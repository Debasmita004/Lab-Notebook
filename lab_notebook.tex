\documentclass{article}
\usepackage[utf8]{inputenc}
\usepackage{geometry}
\usepackage{fancyhdr}
\usepackage{graphicx}
\usepackage{array}
\usepackage{titlesec}
\usepackage{hyperref}
\usepackage{booktabs} 

\geometry{a4paper, margin=1in}
\titleformat{\section}{\large\bfseries}{\thesection}{1em}{}
\titleformat{\subsection}{\normalsize\bfseries}{\thesubsection}{1em}{}

\title{
    \includegraphics[width=0.2\textwidth]{logo.png}\\ 
    \vspace{1cm}
    Lab Notebook
}
\author{Software Tools And Technology}
\date{\today}

\pagestyle{fancy}
\fancyhf{}
\fancyhead[L]{Lab Notebook}
\fancyhead[R]{\thepage}

\begin{document}

\maketitle

\begin{center}
\begin{tabular}{|>{\centering\arraybackslash}m{5cm}|>{\centering\arraybackslash}m{5cm}|>{\centering\arraybackslash}m{5cm}|}
\hline
\textbf{Name} & \textbf{Roll No} & \textbf{Department} \\
\hline
Debasmita Banik & 30001223007 & BCA \\
Anushka Roy & 30059223017 & BSc in Forensic Sciences \\
Joyosmita Ghosh & 30001223077 & BCA \\
Arijit Saha & 30084323007 & BSc in Data Science \\
Jit Sharma & 30001223057 & BCA \\
\hline
\end{tabular}
\end{center}

\newpage

\section*{Lab Notebook Entries}

\subsection*{Debasmita (Lead)}
\begin{itemize}
    \subsection*{Mind Reading Program Using Java:}


    \item Description: 
    
    \subsection*{This Java AWT application is a simple graphical program that simulates a mind-reading trick. The user is prompted to think of any two-digit number, reverse the digits, and find the difference between the original and reversed numbers. The user then finds the resulting number in a grid of symbols, each labeled with a number from 0 to 98.}\\

    \subsection*{The twist of the program is that all numbers divisible by 9 share the same symbol, which is randomly generated each time the program runs. This symbol is eventually revealed as the "mind-read" symbol when the user clicks the "Chin Tapak Dam Dam" button.}\\

    \item Features: 
    \subsection*{Grid of Symbols: The main window displays a grid of 99 symbols, each paired with a number from 0 to 98.}
    \subsection*{Random Special Symbol: A random symbol is assigned to all positions in the grid that are divisible by 9.}
    \subsection*{Instructional Message: The application provides a brief message at the top of the window that guides the user through the mental trick.}
    \subsection*{Submit Button: Once the user is ready, they click the "Chin Tapak Dam Dam" button to reveal the special symbol in a refreshed window.}

\item How It Runs ?

\subsection*{The user is instructed to think of a two-digit number, reverse its digits, and subtract the smaller number from the larger number.}\\

\subsection*{The user then finds the result in the grid of symbols and memorizes the corresponding symbol.}\\

\subsection*{When the user clicks the "Chin Tapak Dam Dam" button, the application clears the grid and displays the special symbol associated with all multiples of 9, "reading the user's mind."}\\ 

\subsection*{For More Details Visit My GitHub Repository:-}\\
\url{https://github.com/Debasmita004/sttdsb}
\end{itemize}

\newpage

\section*{Code For An Mind Reading Program}
\begin{figure}[h!]
    \centering
\includegraphics[width=400,height=600]{image1.jpg}
    \caption{This Program Is Made With Java Which Can Tell Number Which You Are Thinking.}
    \label{fig:image1}
\end{figure}



\newpage
\section*{Lab Notebook Entries}
\subsection*{Anushka Roy}
\subsection{Algorithm for Calculator in C Program}

Below is the algorithm I followed to create a basic calculator in C:

\subsubsection*{Algorithm}

\begin{enumerate}
    \item \textbf{Define Functions:} in C program
    \begin{itemize}
        \item \texttt{add(x, y)}: Returns \texttt{x + y}.
        \item \texttt{subtract(x, y)}: Returns \texttt{x - y}.
        \item \texttt{multiply(x, y)}: Returns \texttt{x * y}.
        \item \texttt{divide(x, y)}: Returns \texttt{x / y} or prints "Cannot divide by zero" if \texttt{y == 0}.
    \end{itemize}
    
    \item \textbf{Main Loop:}
    \begin{itemize}
        \item Display the menu with the following options: Add, Subtract, Multiply, Divide, Exit.
        \item \textbf{Get user choice:}
        \begin{itemize}
            \item If user inputs \texttt{5}, exit the program.
            \item Otherwise, prompt the user to input two numbers.
        \end{itemize}
        \item \textbf{Perform operation based on the user's choice:}
        \begin{itemize}
            \item Call the appropriate function and display the result.
        \end{itemize}
        \item Handle invalid inputs by printing "Invalid input."
    \end{itemize}
\end{enumerate}

\subsubsection*{Example Iteration}

\begin{enumerate}
    \item \textbf{User Chooses Operation:} User enters \texttt{1} for addition.
    \item \textbf{User Inputs Numbers:} User enters \texttt{12} and \texttt{8}.
    \item \textbf{Result Calculation:} \texttt{add(12, 8)} returns \texttt{20}.
    \item \textbf{Output:} Result is \texttt{20.0}.
    \item \textbf{User Chooses to Exit:} User enters \texttt{5}.
    \item \textbf{Program Ends.}
\end{enumerate}

\href{https://github.com/Anushka2127/calculator_anushka}{CLICK: GitHub Repo for Calculator}

\newpage
\section{Joyosmita Ghosh}

\subsection{Algorithm for Calculator in C Program}

Below is the algorithm I followed to create a basic calculator in C:

\subsubsection*{Algorithm}

\begin{enumerate}
    \item \textbf{Define Functions:} in C program
    \begin{itemize}
        \item \texttt{add(x, y)}: Returns \texttt{x + y}.
        \item \texttt{subtract(x, y)}: Returns \texttt{x - y}.
        \item \texttt{multiply(x, y)}: Returns \texttt{x * y}.
        \item \texttt{divide(x, y)}: Returns \texttt{x / y} or prints "Cannot divide by zero" if \texttt{y == 0}.
    \end{itemize}
    
    \item \textbf{Main Loop:}
    \begin{itemize}
        \item Display the menu with the following options: Add, Subtract, Multiply, Divide, Exit.
        \item \textbf{Get user choice:}
        \begin{itemize}
            \item If user inputs \texttt{5}, exit the program.
            \item Otherwise, prompt the user to input two numbers.
        \end{itemize}
        \item \textbf{Perform operation based on the user's choice:}
        \begin{itemize}
            \item Call the appropriate function and display the result.
        \end{itemize}
        \item Handle invalid inputs by printing "Invalid input."
    \end{itemize}
\end{enumerate}

\subsubsection*{Example Iteration}

\begin{enumerate}
    \item \textbf{User Chooses Operation:} User enters \texttt{1} for addition.
    \item \textbf{User Inputs Numbers:} User enters \texttt{12} and \texttt{8}.
    \item \textbf{Result Calculation:} \texttt{add(12, 8)} returns \texttt{20}.
    \item \textbf{Output:} Result is \texttt{20.0}.
    \item \textbf{User Chooses to Exit:} User enters \texttt{5}.
    \item \textbf{Program Ends.}
\end{enumerate}

\href{https://github.com/joyosmita/calculator}{CLICK: GitHub Repo for Calculator}


\newpage
\newpage
\section*{Lab Notebook Entries}
\subsection*{Arijit Saha}

\subsection*{Mind Reading Program Using Java:}

\begin{itemize}
    \item \textbf{Description:} This Java AWT application is a simple graphical program that simulates a mind-reading trick. The user is prompted to think of any two-digit number, reverse the digits, and find the difference between the original and reversed numbers. The user then finds the resulting number in a grid of symbols, each labeled with a number from 0 to 98.\\

    The twist of the program is that all numbers divisible by 9 share the same symbol, which is randomly generated each time the program runs. This symbol is eventually revealed as the "mind-read" symbol when the user clicks the "Chin Tapak Dam Dam" button.\\

    \item \textbf{Features:}
    \begin{itemize}
        \item \textbf{Grid of Symbols:} The main window displays a grid of 99 symbols, each paired with a number from 0 to 98.
        \item \textbf{Random Special Symbol:} A random symbol is assigned to all positions in the grid that are divisible by 9.
        \item \textbf{Instructional Message:} The application provides a brief message at the top of the window that guides the user through the mental trick.
        \item \textbf{Submit Button:} Once the user is ready, they click the "Chin Tapak Dam Dam" button to reveal the special symbol in a refreshed window.
    \end{itemize}
    
    \item \textbf{How It Runs?}
    The user is instructed to think of a two-digit number, reverse its digits, and subtract the smaller number from the larger number.\\

    The user then finds the result in the grid of symbols and memorizes the corresponding symbol.\\

    When the user clicks the "Chin Tapak Dam Dam" button, the application clears the grid and displays the special symbol associated with all multiples of 9, "reading the user's mind."\\ 

    \item \textbf{For More Details Visit My GitHub Repository:} \url{https://github.com/arijit22568/STT}
\end{itemize}
\newpage
\section{Jit Sharma}

\subsection{Algorithm for Calculator in C Program}

Below is the algorithm I followed to create a basic calculator in C:

\subsubsection*{Algorithm}

\begin{enumerate}
    \item \textbf{Define Functions:} in C program
    \begin{itemize}
        \item \texttt{add(x, y)}: Returns \texttt{x + y}.
        \item \texttt{subtract(x, y)}: Returns \texttt{x - y}.
        \item \texttt{multiply(x, y)}: Returns \texttt{x * y}.
        \item \texttt{divide(x, y)}: Returns \texttt{x / y} or prints "Cannot divide by zero" if \texttt{y == 0}.
    \end{itemize}
    
    \item \textbf{Main Loop:}
    \begin{itemize}
        \item Display the menu with the following options: Add, Subtract, Multiply, Divide, Exit.
        \item \textbf{Get user choice:}
        \begin{itemize}
            \item If user inputs \texttt{5}, exit the program.
            \item Otherwise, prompt the user to input two numbers.
        \end{itemize}
        \item \textbf{Perform operation based on the user's choice:}
        \begin{itemize}
            \item Call the appropriate function and display the result.
        \end{itemize}
        \item Handle invalid inputs by printing "Invalid input."
    \end{itemize}
\end{enumerate}

\subsubsection*{Example Iteration}

\begin{enumerate}
    \item \textbf{User Chooses Operation:} User enters \texttt{1} for addition.
    \item \textbf{User Inputs Numbers:} User enters \texttt{12} and \texttt{8}.
    \item \textbf{Result Calculation:} \texttt{add(12, 8)} returns \texttt{20}.
    \item \textbf{Output:} Result is \texttt{20.0}.
    \item \textbf{User Chooses to Exit:} User enters \texttt{5}.
    \item \textbf{Program Ends.}
\end{enumerate}

\href{https://github.com/jitsharma2003/create-calculator.c}{CLICK: GitHub Repo for Calculator}


\newpage

\newpage


\newpage

\section*{Conclusion And Acknowledgement}

\subsection*{

---

We would like to extend our heartfelt gratitude to you for your unwavering guidance and support throughout the creation of our group of five members. Your insights and encouragement have been instrumental in fostering a collaborative and productive environment. \\

From the initial stages of forming our group to the final touches on our LaTeX lab notebook, your advice has been invaluable. You have not only provided us with the technical knowledge required but also inspired us to work cohesively as a team. This experience has significantly enhanced our understanding and skills, and we are deeply appreciative of the opportunity to learn under your mentorship.\\

We are pleased to inform you that our LaTeX lab notebook is now perfectly ready. We have meticulously compiled and formatted all our findings and experiments, ensuring that the notebook meets the highest standards of accuracy and presentation. Additionally, we have successfully added the completed notebook to the GitHub repository, making it accessible for future reference and collaboration.\\

Thank you once again for your unwavering support and for fostering an environment that allowed us to collaborate effectively and achieve our goals. Your dedication to our learning and development has made a profound impact on us, and we are truly grateful for your mentorship.\\

---

}



\end{document}
